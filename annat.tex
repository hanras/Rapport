Bakgrund

Finanskrisen 2007-2008 och det faktum att den inte förutsågs och begränsades visade tydligt svagheten hos ekonomiska standardmodeller. Dessa är baserade på rationella välinformerade aktörer som gör att marknaden aldrig avviker långt från jämvikt, medan verkligheten innehåller komplikationer såsom nätverk av sammankopplade strukturer, begränsad information, positiva återkopplingsmekanismer och flockbeteenden.


Problembesrkivning

Vi begränsar oss till aktiemarknaden där klassiska modeller förutsätter att aktiekurser i huvudsak följer en slumpvandring, vilket ger upphov till en normalfördelning av kursförändringarna. Det har visat sig att i den verkliga utvecklingen av aktiekurser är det betydligt vanligare med stora fluktuationer än förväntad från en normalfördelning. En metod som använts (inom området som kallats Econophysics) är agentbaserad modellering, där enkla beteenden hos samverkande individer på mikronivå kan ge upphov till komplexa fenomen på makronivå. Här kan t.ex. gruppbeteenden ge stora prisfluktuationer.